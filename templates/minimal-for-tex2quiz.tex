\documentclass{article}

%% For inparaenum.
\usepackage{paralist}

%% If there are no selected question, show everything.
\newif\ifnoqnum
\ifcsname qnum\endcsname
  \usepackage[active, tightpage]{preview}
  \setlength\PreviewBorder{0pt}%
  \noqnumfalse
\else
  \usepackage{preview}
  \noqnumtrue
\fi

%!EXTRAPREAMBLE

\begin{document}
\begin{enumerate}
\begin{preview}

%!BEGIN_QUESTIONS
def isAttrTrue (elt, field):
    return elt.get (field, "false") != "false"

## We count the number of questions, and if it matches \qnum, print it.
global nquestion
if not 'nquestion' in globals ():
  nquestion = 0
nquestion += 1

## Bypass the test if no \qnum were given, to print everything.
print ("\\ifnoqnum\\gdef\\qnum{" + str (nquestion) + "}\\fi")
## This if holds iff qnum = nquestion
print ("\\ifnum\qnum=" + str (nquestion) + "")

## Same as minimal.tex
print ("\\item " + question.text + "\n\n")
if not isAttrTrue (question, "hideanswers"):
  if isAttrTrue (question, "onepar"):
    env = "inparaenum"
  else:
    env = "enumerate"
  print ("\\begin{" + env + "}[A.]")
  for ans in answers:
    print ("\\item " + ans.text)
    if isAttrTrue (ans, "correct"):
      print (" (correct)")
  print ("\\end{" + env + "}")
points = question.get ("points", "1")
print ("\\hfill (" + points + "pt" + \
                ("s" if int (points) > 1 else "") + ")\n")

print ("\\fi")
%!END_QUESTIONS

\end{preview}
\end{enumerate}
\end{document}
